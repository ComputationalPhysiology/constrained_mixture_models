\documentclass[twoside,12pt,a4paper]{article}
\usepackage{exscale,times}
\usepackage{graphicx}
\usepackage{amsmath}


\title{Homogenized constrained mixture models for cardiac growth and remodeling}
\author{Joakim Sundnes}

\begin{document}

\maketitle

\section{Background}
\begin{itemize}
  \item We want to explore constrained mixture models for cardiac growth and remodeling. 
  \item The framework of homogenized constrained mixture (HCM) models \cite{cyron2016homogenized,braeu2017homogenized} 
        appears to be a good starting point. It can be viewed as a compromise between the constrained mixture models
        and the volumetric growth framework introduced by \cite{rodriguez1994stress}.
  \item Gebauer et al \cite{gebauer2022homogenized} present a homogenized constrained mixture models for cardiac G{\&}R, and apply 
        it to an idealized left ventricle.
  \item To get a better understanding of the model framework, we here present a non-growing version of the mixture model, which
        has also been implemented and solved in FEniCS. 
\end{itemize}

\section{A non-growing constrained mixture model}
We first consider the simplified case with no growth, where the govering equation is 
a standard quasi-static equilibrium model for large deformations:
\[
\nabla\cdot P = 0,  
\]
Here, $P$ is the second Piola-Kirchhoff stress tensor, given by 
\[
P = \frac{\partial \Psi}{\partial F},  
\]
with $F$ being the deformation gradient and $\Psi$ a given strain energy, defined per unit volume. The material is modeled as a constrained mixture, i.e., a 
mixture of $n$ constituents which are constrained to deform together, eliminating the need to model
momentum exchange between the constituents. The strain energy is then given by
\[
\Psi = \sum_{i=0}^n \rho_0^i W^i(C_e^i) + \Psi_{vol} , 
\]
where $W_i$ is the strain energy per unit mass for constituent $i$, $\rho_0^i$ is the
reference mass density of constituent $i$, and $C_e^i = (F_e^i)^TF_e^i$ is the Cauchy-Green
deformation tensor of the elastic deformation of constituent $i$. For now, since we
assume no growth, we have $F_e^i = F^i = F$, since the entire deformation is elastic and
all constituents deform together. The strain energy therefore simplifies to 
\[
\Psi = \sum_{i=0}^n \rho_0^i W^i(C) + \Psi_{vol} , 
\]
with $C = F^T F$. We assume that the myocardium consists of three load bearing constituents: 
collagen ($c$), myocytes ($m$), and a "background" constituent consisting mainly of elastin ($b$).
For now, we include a single family of collagen fibers, aligned in the same direction as the
myocytes. The background consitituent is assumed to be isotropic. Following 
Holzapfel-Ogden \cite{holzapfel2009constitutive}and
Gebauer et al \cite{gebauer2022homogenized}, we adopt the following forms of the individual strain energies:
\begin{align*}
  W^b(C) &= \frac{a}{2b}\exp[b (\bar{I}_1 -3)],  \\
  W^c(C)  &= \frac{a_c}{2b_c}\exp[b_c (I_{4c} -1)^2], \\
  W^m(C) &= W_{pas}^m + W_{act}^m, \\
  W_{pas}^m &= \frac{a_m}{2b_m}\exp[b_m (I_{4m} -1)^2], \\ 
  W_{act}^m &= S_a(\tau) \sqrt{I_4m} \\
  \Psi_{vol} &= \frac{\kappa}{2}\left(J -\frac{\rho_0(s)}{\rho_0(s=0)} \right)^2
\end{align*}
Here, $\bar{I}_1$ is the modified first invariant of the Cauchy-Green tensor $C$, and $I_{4c},I_{4m}$ are
the fourth pseudo-invariants defined by $I_{4c} = f_r^c C f_r^c, I_{4m} = f_r^m C f_r^m$, with $f_r^c, f_r^c$ 
being the preferred direction of the fiber families. For now, we assume $f_r^c = f_r^m = f_0$, so 
we have $I_{4c} = I_{4m}$. The myocyte strain energy has been split into active and passive parts, following
(Gebauer et al 2022) and (Pezzuto and Ambrosi 2010). All material parameters are 
chosen as in (Gebauer et al 2022). Since, for now, we assume no growth of the tissue, the 
reference density $\rho_0$ is constant, and the volumetric energy term simplifies
to 
\[ 
  \Psi_{vol} = \frac{\kappa}{2}\left(J - 1 \right)^2 ,
\]
with $J= \det(F)$ and $\kappa$ being a penalty parameter (bulk modulus).

The model above has been implemented in FEniCS and, as expected, its behavior is similar to the standard Holzapfel-Ogden model. 
The only advantages of the present formulation are connected to the interpretation of the model terms. We can intuitively
change the tissue composition by changing the mass fraction, and it is easy to add collagen fiber families in arbitrary 
directions. The same changes can be achieved in the Holzapfel-Ogden model by modifying the material parameters, but the physical 
interpretation is less obvious. However, any real value of the models comes when adding growth and remodeling laws for 
individual constituents, to allow the tissue composition to change in response to mechanical stimuli. This part is complicated
by the fact that we cannot only model changes in mass fractions, but also need to consider prestretch tensors associated with 
the growth and remodeling of each constituent, similar to the growth tensors applied in the volumetric 
growth framework from Rodriguez et al \cite{rodriguez1994stress}.

\subsection{Growth and remodeling of the mixture}
In the original constrained mixture models \cite{humphrey2002constrained}, each constituent is continuously produced and 
degraded. Since new material is added with a given prestretch, this turnover can lead to a change in the tissue configuration
even if the net mass production is zero. To compute the current elastic deformation of the tissue, which determines the stress 
and thereby the stimulus for new growth, we need to keep track of the prestretch of all added material at all previous times. 
The impact of the prestretch of material added at time $\tau$ decays exponentially with time, and we can therefore set a cutoff
time for how long we need to track the prestretch. In a discretized model, this bookkeeping involves 
storing a prestretch tensor field for each constituent for a finite number of previous time steps. Although entirely manageable, 
this complicates the computational model considerably, and motivated the introduction of a temporally homogenized 
constrained mixture (HCM) model \cite{cyron2016homogenized}, where the influence of all previous material turnover is lumped into a single
prestretch tensor. This framework appears to be a good compromise between the complexity of the original constrained mixture
models and the simpler framework of volumetric growth \cite{rodriguez1994stress}.

Gebauer et al \cite{gebauer2022homogenized} presents a relatively simple HCM model for cardiac tissue, to describe growth and remodeling of an 
idealized left ventricle. The model includes the constituents listed above (myocytes, collagen, elastin), with similar
formulations of the strain energies except for a different formulation of the active tension. 
Myocytes and collagen are assumed to remodel over time to maintain a given stress homeostasis, with net mass production
governed by the following law:
\[
\dot{\rho_0^i} = \rho_0^ik_\sigma^i \frac{\sigma^i-\sigma_h^i}{\sigma_h^i} . 
\]
Here, $\rho_0^i$ is the reference density of constituent $i$ (changes with time since mass is added while the reference volume 
is unchanged), $\sigma^i$ is the Cauchy stress in the constituent's preferred direction, $\sigma_h^i$ is the homeostatic stress,
and $k_\sigma^i$ is a growth gain factor. We also need to describe the degradation of constituent $i$, given by
\[
  \dot{d\rho_{0-}^i} = -\frac{\rho_0^i}{T^i},  
\]
where $T^i$ is the mean lifetime of deposited mass, and the mass production 
\[
  \dot{d\rho_{0+}^i} = \dot{\rho_0^i} - \dot{d\rho_{0-}^i} .
\]

Mocytes and collagen are modeled as one-dimensional fiber families, and in \cite{gebauer2022homogenized} 
it is assumed incompressible remodeling and that all new mass is deposited in the fiber direction. The remodeling (prestretch) deformation tensor can 
then be written as
\[
F_r^i = \lambda_r^i f_0^i \otimes f_0^i  +\frac{1}{\sqrt{\lambda_r^i}}(I-f_0^i \otimes f_0^i),
\]
where $\lambda_r^i$ is the remodeling stretch in the fiber direction, and is governed by an ODE (Eq (9) in Gebauer et al (2022)).

Elastin is not assumed to remodel in response to mechanical stimuli, but an isochoric and rotation-free prestretch tensor is 
still applied (?).

\subsection{Motivation, project ideas and open questions}
Why should we consider these models at all?
\begin{itemize}
  \item Novelty. The constrained mixture models were proposed for vascular tissue, and there's only one publication applying
        them to cardiac tissue \cite{gebauer2022homogenized}. Almost any implementation and application of the model in a cardiac 
        setting will be novel and publishable.
  \item Complexity. Complexity and solver efficiency are probably the main obstacles towards more widespread of the models. 
        Simula is well positioned to tackle this.
  \item Fibrosis and tissue stiffness. Although fairly simplified, the models provide a mechanistic model framework for 
        describing fibrosis in response to mechanical load or other stimuli, and the effect on the mechanical properties.
  \item Open questions. The growth stimulus in \cite{gebauer2022homogenized} appears to be quite simple, and may be more applicable 
        for vascular tissue than cardiac. Combining the HCM model framework with cardiac growth laws as proposed in
        \cite{kerckhoffs2012single} or \cite{lee2016integrated} could probably form the basis for both methods- and application papers.
  \item Comparison with the volumetric growth model and possibly the non-homogenized constrained mixture models. The study 
        by Witzenburg and Holmes \cite{witzenburg2017comparison} could be a good framework for a comparison. They test and compare various growth
        laws within the volumetric growth framework, and it could be interesting to do something similar using different
        versions of constrained mixture models. 
\end{itemize}

The overall idea of the HCM model framework is quite clear from \cite{cyron2016homogenized,gebauer2022homogenized}. 
However, some details are not fully clear (to me) from the description in the papers, and may require diving a bit 
deeper into some of the cited references:
\begin{itemize}
  \item The description of the prestretch tensor for the elastin constituent is not fully clear. If elastin does not turn over,
        I would intuitively expect this constituent to be in a constant state of stretch as the other consituents grow. Instead,
        the authors adopt an iterative procedure introduced in \cite{mousavi2017patient} to make this constituent stress free (?).
  \item What is the reasoning behind the isochoric remodeling of the fibers. As fibers grow longer in response to stretch, why 
        would they also be thinner? This may be related to the model formulation, i.e., that the actual volume change is handled
        elsewhere, or it could be a simplifying assumption. If the latter is the case it should probably be changed, 
        in particular for the remodeling of the myocytes.
  \item The model above, adopted from \cite{gebauer2022homogenized}, includes a single family of collagen fibers. This is 
        probably ok for now, but may have to be extended later. Not clear how many fiber families it is natural to include? 
        Maybe two, aligned in the fiber- and sheet direction?
\end{itemize}

Possible first steps:
\begin{itemize}
  \item Since the unit cube model presented above is completely homogenous, it can be simplified to a 0D balance equation which 
        does not require solving with the finite element method. A 0D model may be a good starting point for introducing 
        growth in the model and explore some initial model choices.
  \item When the 0D model is working, implement a 3D unit cube model with growth and remodeling, and verify that the two 
        give the same results.
  \item Test the model predictions for the simple case proposed in \cite{witzenburg2017comparison}? Possible publication.
  \item Explore growth laws and mechanical quantities that drive growth, consider for instance \cite{lee2016integrated}.
  \item Expand to more realistic geometries. 
\end{itemize}





\bibliographystyle{plain}
\bibliography{./growth_remodeling}















\end{document}
