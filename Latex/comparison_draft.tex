\documentclass[twoside,12pt,a4paper]{article}
\usepackage{exscale,times}
\usepackage{graphicx}
\usepackage{amsmath}


\title{A comparison of models for cardiac growth and remodeling}
\author{Joakim Sundnes}

\begin{document}

\maketitle

\section{Introduction}
Soft tissue growth and remodeling has been modeled in a number of different frameworks,
including for instance:
\begin{itemize}
  \item The volumetric growth model introduced by Rodriguez et al (1994).
  \item Constrained mixture models, introduced by Humphrey and Rajagopal (2002).
  \item Homogenized constrained mixture models, proposed by Cyron et al (2016).
\end{itemize}
These frameworks vary considerably in terms of computational complexity and (potential) 
biological detail. We will here compare the different models for the simple case of a 
homogeneous unit cube, to explore the differences in more detail.


\section{Mechanical model problem}
The starting point is a standard, quasi-static mechanics model of active cardiac tissue:
\begin{align}
\nabla\cdot(FS) &= 0,  \label{elasticity_1} \\
S &= S^p + S^a,  \label{stress_split} \\
%\Psi &= \frac{1}{2}K(e^W-1) \label{strain_energy1} \\
%W &= b_{ff}E_{11}^2+b_{xx}(E_{22}^2+E_{33}^2+E_{23}^2)
%+b_{fx}(2E_{12}^2+2E_{13}^2). \label{strain_energy2} \\
S^p &= {\partial\Psi\over\partial E}+pC^{-1}, \\
S^a &= J F^{-1}\sigma^a(s,\lambda,\dot{\lambda})F^{-T}, \label{active_s}
\end{align}
Here, \eqref{elasticity_1} describes mechanical equilibrium, 
with $F$ being the deformation gradient and $S$ the the second Piola-Kirchoff
stress tensor, which is defined by constitutive relations \eqref{stress_split}-\eqref{active_s}.
Furthermore, $J=det(F)$, (required to satisfy $J=1$ because of incompressibility), 
$W$ is the strain energy function which defines the passive material properties of the 
tissue, $C$ is the right Cauchy-Green tensor, $p$ is the hydrostatic pressure, 
%$E$ is the Green-Lagrange strain tensor, 
and $\sigma^a$ is the active part of the Cauchy stress tensor. In
local fiber coordinates we have $\sigma^a = \mbox{diag}(T_a,\eta
T_a, \eta T_a)$, where $\eta$ is a constant and $T_a$ is a prescribed active tension. 
In realistic models $T_a$ depends on the deformation state, but for simplicity we will first 
assume that it is a prescribed function of time only. 

The equations must be complemented with appropriate boundary conditions. We want to derive 
the simplest possible model, and consider a mechanics problem defined on the unit cube. 
We apply the following combination of Dirichlet and Neumann boundary conditions
\begin{align*}
u_X & = 0 \mbox{ for  } X = 0, \\
FS \cdot n &= \boldsymbol{T} \mbox{ for  } X = 1, \\
FS \cdot n &= 0 \mbox{ for  } Y = 0, Y = 1, Z = 0 \mbox{ or } Z = 1, \\
\end{align*}
plus additional necessary constraints to avoid rigid body motion. Here $u$ is the displacement, 
$u_X$ is the displacement in the $X$-direction, and $X,Y,Z$ are the coordinates of the undeformed
reference configuration. These conditions define a cube that is held fixed at one end, loaded with 
a normal load $T$ at the other end, and unloaded in the transverse directions. 

\subsection{A mixture-based material model}
It remains to specify 









If we further assume that the cube is homogeneous, with fibers aligned in the $X$-direction, and
that the prescribed active stress is homogeneous, the boundary conditions above will lead to a 
uniform deformation field with a diagonal deformation gradient with components
$F_{11} = \lambda, F_{22}=F_{33} = \sqrt{1/\lambda}$.
The right Cauchy-Green tensor $C$ and the Green-Lagrange tensor
$E$ will also be diagonal, with components $C_{11} = \lambda^2 , 
C_{22} =C_{33} = 1/\lambda,  
E_{11} = \frac{1}{2}(\lambda^2 - 1)$, and 
$E_{22} = E_{33} = \frac{1}{2}(\frac{1}{\lambda}-1).$\\

Inserting these expressions into the strain energy function above
defines the passive stress in terms of $lambda,p$;
\begin{align}
S_{11} &= \frac{\partial\Psi}{\partial E_{11}}+p\{ C^{-1}\}_{11}  =
\frac{1}{2}Kb_{ff}(\lambda^2-1)e^W + p\frac{1}{\lambda^2} \nonumber\\
S_{22} = S_{33} &=  \frac{\partial\Psi}{\partial E_{22}}+p\{ C^{-1}\}_{22} 
= \frac{1}{2}K b_{xx}(1/\lambda -1)e^W + p\lambda .\nonumber  
\end{align}
Since the slab is unloaded, and we have disregarded body forces and inertia effects, 
the active and passive stresses must balance in all points. Eq \eqref{elasticity_1}
may therefore be replaced by $S^a + S^p = 0.$ 

\section{Growth and remodeling of the unit cube}
In this section we will extend the model from above 




\end{document}
