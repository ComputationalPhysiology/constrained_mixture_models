%------------------------------------------------------------------------------
\documentclass[a4paper,10pt]{article}
%------------------------------------------------------------------------------

\usepackage{amssymb}
\usepackage{amsmath}
\usepackage{graphicx}

% Some math operators
\DeclareMathOperator{\Div}{div}
\DeclareMathOperator{\Grad}{grad}
\newcommand{\inner}[2]{\langle #1, #2 \rangle}
\newcommand{\R}{\mathbb{R}}
\newcommand{\foralls}{\forall\,}
\newcommand{\ddt}[1]{{#1}_t}
\newcommand{\dx}{\dif{}x}
\newcommand{\ds}{\dif{}s}
\newcommand{\dS}{\dif{}S}
\newcommand{\tr}{\mbox{tr}}
\newcommand{\eps}{\epsilon}

\begin{document}
\section{Growth laws formulated as ODEs}
It is common to write growth laws in the form of one or more ODEs, where growth deformation is driven by a deviation
from a set point. Considering the simple case of isotropic volume growth, the growth tensor $F_g$ can be written as
\begin{equation}
F_g = \left(\frac{V}{V_0}\right)^{1/3} I,
\label{growth0}\end{equation}
where $V_0$ and $V$ are local tissue volumes before and after growth, respectively, and $I$ is the identity. 
A growth law describing the volume change can look like 
\begin{equation}
\frac{dV}{dt} = \beta V (s-s_{set}) ,
\label{ODE0}\end{equation}
where $\beta$ is a constant (growth rate factor), $s$ is the quantity driving the growth, typically computed from 
local strain or stress, $s_{set}$ is the homeostatic set point for this quantity, and $s-s_{set}$ represents
the growth stimulus \cite{Kroon2009BMM}. 

More complex growth laws derived to describe cardiac growth are often formulated in terms of algebraic 
relations relating an incremental growth tensor directly to the stimulus, bypassing the ODE formulation above. 
One relevant example is the growth law from \cite{Kerckhoffs2012MRC}, which defines the
incremental growth tensor as a diagonal tensor in the fiber coordinate system;
$F_{g,i}= diag(F_{g,i,ff}, F_{g,i,cross}, F_{g,i,cross})$, with the components 
$F_{g,i,ff}, F_{g,i,cross}$ defined by eqs (8) and (9) in \cite{Kerckhoffs2012MRC}. 

We want to explore the relation between the two formulations of growth laws, and in particular
to determine if the incremental growth law in \cite{Kerckhoffs2012MRC} can be derived 
from an ODE formulation such as \eqref{ODE0}. To do this, we first consider a simple forward
discretization of \eqref{ODE0}:
\[
\frac{V_{n+1} - V_{n}}{\Delta t} = \beta V_n (s_n-s_{set}),
\]
where $V_n, s_n$ are the volume and the stimulus, respectively, at time step $n$. From this, we
get
\[
V_{n+1} = V_n (1 + \Delta t \beta (s_n-s_{set}))
\]
and we can insert into \eqref{growth0} to get
\[
F_{g,n+1} = \left(\frac{V_n}{V_0}(1 + \Delta t \beta (s_n-s_{set}))\right)^{1/3} I = F_{g,i} F_{g,n},
\]
where we have defined $F_{g,n}$ as the growth tensor at step $n$, and the incremental growth tensor $F_{g,i}$ as
\[
  F_{g,i} = (1 + \Delta t \beta (s_n-s_{set})) I .
\]

Now, consider a more general version of the growth tensor in \eqref{growth0}, describing transversely isotropic
growth; $F_{g}= diag(F_{g,ff}, F_{g,cross}, F_{g,cross})$. 
The individual components $F_{g,ff}, F_{g,cross}$ are governed by ODEs
\begin{align}
  \frac{d F_{g,ff}}{dt} &= \begin{cases}
    k_\text{ff} F_{g,ff} \frac{f_\text{ff,max}}{1 + \exp(-f_f(s_l - s_{l,50}))}, & s_l \geq 0 \\
    k_\text{ff} F_{g,ff} \frac{-f_\text{ff,max}}{1 + \exp(f_f(s_l + s_{l,50}))}, & s_l < 0
    \end{cases} \label{kerckhoffs_ode0}\\
    \frac{d F_{g,cross}}{dt} &=
\begin{cases}
  F_{g,cross} \sqrt{k_\text{cc}\frac{f_\text{cc,max} }{1 + \exp(-c_f(s_t - s_{t,50}))}}, & s_t \geq 0 \\
  F_{g,cross} \sqrt{k_\text{cc} \frac{-f_\text{cc,max}}{1 + \exp(c_f(s_t + s_{t,50}))}}, & s_t < 0
\end{cases}, \label{kerckhoffs_ode1}
\end{align}
where the right-hand sides are adapted from \cite{Kerckhoffs2012MRC}.
Here, $s_l, s_t$ are
stimuli for growth in the longitudinal and transverse direction, respectively. Both are computed
as deviations from given set points (see eqs (5)-(7) in \cite{Kerckhoffs2012MRC}), similar to the 
stimulus $s-s_{set}$ in the model above. The scaling factors
$k_\text{ff},k_\text{cc}$ are defined in (11)-(12) in \cite{Kerckhoffs2012MRC}), and are effectively 
enforcing hard limits on the cumulative tissue growth. 

We can discretize \eqref{kerckhoffs_ode0}-\eqref{kerckhoffs_ode1} with an explicit scheme, in the same way as we did for the
simpler growth law, to get
\begin{align}
  \frac{F_{g,ff}^{n+1} - F_{g,ff}^{n}}{\Delta t} &= \begin{cases}
    k_\text{ff} F_{g,ff}^n \frac{f_\text{ff,max} }{1 + \exp(-f_f(s_l^n - s_{l,50}))}, & s_l \geq 0 \\
    k_\text{ff} F_{g,ff}^n \frac{-f_\text{ff,max} }{1 + \exp(f_f(s_l^n + s_{l,50}))}, & s_l < 0
    \end{cases} \label{kerckhoffs_disc0}\\
    \frac{F_{g,cross}^{n+1}- F_{g,cross}^n}{\Delta t} &=
\begin{cases}
  F_{g,cross}^n \sqrt{k_\text{cc} \frac{f_\text{cc,max}  }{1 + \exp(-c_f(s_t^n - s_{t,50}))}}, & s_t \geq 0 \\
  F_{g,cross}^n \sqrt{k_\text{cc} \frac{-f_\text{cc,max} }{1 + \exp(c_f(s_t^n + s_{t,50}))}}, & s_t < 0
\end{cases}, \label{kerckhoffs_disc1}
\end{align}
and then
\begin{align}
  F_{g,ff}^{n+1} &= \begin{cases}
    F_{g,ff}^n (k_\text{ff} \frac{f_\text{ff,max} \Delta t}{1 + \exp(-f_f(s_l^n - s_{l,50}))} + 1), & s_l \geq 0 \\
    F_{g,ff}^n (k_\text{ff} \frac{-f_\text{ff,max} \Delta t}{1 + \exp(f_f(s_l^n + s_{l,50}))} + 1), & s_l < 0
    \end{cases} \label{kerckhoffs_inc0}\\
    F_{g,cross}^{n+1} &=
\begin{cases}
  F_{g,cross}^n (\sqrt{k_\text{cc} \frac{f_\text{cc,max} \Delta t }{1 + \exp(-c_f(s_t^n - s_{t,50}))}}+1), & s_t \geq 0 \\
  F_{g,cross}^n (\sqrt{k_\text{cc}\frac{-f_\text{cc,max}\Delta t }{1 + \exp(c_f(s_t^n + s_{t,50}))}}+ 1), & s_t < 0
\end{cases}. \label{kerckhoffs_inc1}
\end{align}
This last formulation can be written as
\begin{align}
  F_{g,ff}^{n+1} &= F_{g,ff}^n F_{g,ff,i},\\
  F_{g,cross}^{n+1} &= F_{g,cross}^n F_{g,cross,i},
\end{align}
with the incremental growth tensors $F_{g,ff,i}, F_{g,cross,i}$ defined as in \cite{Kerckhoffs2012MRC}. 

\bibliographystyle{plain}
\bibliography{./references}

\end{document}